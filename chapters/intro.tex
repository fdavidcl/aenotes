\setchapterpreamble[u]{\margintoc}
\chapter{Introduction}
\labch{intro}


The current trends in collection of data are increasing, with more and more human activities producing machine-readable information, such as product reviews, posts on social media, medical images, industrial machinery sensor data, \xr{más ejemplos y citas} and more. Automatic processing of data makes it easier to obtain  results fast and saves hours of human labor which can be freed for other purposes or dedicated to tasks which cannot be automated. The speed provided by current computation resources also opens new possibilities for leveraging the available data, achieving extraction and manipulation of information at levels infeasible by human hand.

The study of problems, tools and solutions related to data integration, processing and analysis has been recently known as data science \xr{cita}, a field which overlaps branches of mathematics, statistics and computer science, among other disciplines. Current data science applications are present everywhere, from the most industrial settings to direct final user access, and range from machinery fault detection, to medical diagnosis assistance, customer loyalty in retail and photograph enhancing \xr{citas}.

The general objective in a data science problem is to model a real world scenario based on the collected data and use the resulting model to provide some information to the end user, for example, a category or label, a ranking, an association or a transformed version of the original data. This is a process that encompasses all steps from data acquisition, to its preparation, processing and analysis.

\section{Problem setting}

A great part of the time spent in a data science problem, usually the longest, consists in preparing and preprocessing the available data in order for the posterior learning techniques to extract the maximum possible amount of information \xr{cita preprocessing}. There exist several traits of the data that can be manipulated to facilitate the work of learning algorithms: missing values, noisy instances, class imbalance, among others.

One aspect of data to which machine learning \xr{definition?} models may be specially sensitive is the set of features, the specific representation of each instance.

- what difficulties can the feature set present

Features that may characterize an event or object adequately for humans may not be ideal for machines to process. For example, a string of text may have meaning for a reader but for the machine it is just a sequence of characters. Furthermore, the techniques used for collecting data can only produce the ``observable'' variables in a dataset, but there may be interesting, ``hidden'' variables which influence the data in a clearer way.

- what problems cause the feature set on classifiers

Providing a learning algorithm with unprocessed features usually leads to sub-par performance due to 

- what is the objective, to get a better feature set

All this leads to a new task that can be performed before the actual knowledge extraction, \textit{feature learning}. The objective when learning features is 

- what are the difficulties of obtaining a new feature set

- what are other possible difficulties: difficult classes and nonstandard problems

Other possible obstacles that one may come across when inspecting the features of a dataset are difficult classes, also known as class complexity. This scenario arises when the variables do not provide sufficient information to allow class separation, the relations between variables and the target are highly nonlinear, or other circumstances prevent a learning method from inferring an adequate mapping from the input features to the target variable.

\section{Tools}

- Qué es el aprendizaje profundo y cómo surge

- Por qué aplicar aprendizaje profundo



\section{Motivation}

The questions that we are trying to tackle throughout this thesis can be summarized as follows:

\begin{itemize}
    \item How does one approach representation learning with deep neural models?
    \item What benefits can one obtain by transforming data into an appropriate representation?
    \item Can one induce specific behavior within the transformations, such as separating different classes?
\end{itemize}


- the potential of deep neural models over other methods for feature learning

As the trends in usage of deep learning models to solve machine learning problems continue to increase, we focus our interest in their potential to not only tackle supervised problems such as classification, regression or detection, but also wider problems where solutions are not so easily validated, such as feature learning. Since deep learning allows the integration of the feature extraction stage directly within the predictor itself, these types of models should be valuable feature learners for other tasks as well.

- interest in how to adapt feature generation for different purposes, e.g. multiple outputs

Deep neural models dedicated to generating new feature sets could be adapted, as a result, to different purposes. For instance, one could search for feature spaces where ``ordinary'' data points are very cohesive, and thus anomalous inputs would be easy to identify. Similarly, a transformation of features could allow for better separation of different classes, better distinction between noise and signal, or more meaningful traits that relate to all the original variables.

- interest in multi-view problems

Another potential use of feature learning models that catches our interest is the possibility of capturing more than one aspect of each problem instance, which translates to different \textit{views} of the problem, for example, image and text. These views could be processed by different learners or special algorithms, but one could build feature learners that combine the available information into a more machine-ready feature vector.

- interest in interpretable models

- 

\section{Objective}

- explore ways of learning feature sets by means of deep learning models

- develop new objectives so that features are able to separate different classes better

\section{Thesis structure}

The rest of this document contains all the needed information to develop the posed questions and provide enough context in order to understand the work that has been realized throuhgout the thesis.

\begin{enumerate}
    \item Chapter~\ref{ch:theory} describes the basic theory that lies under the later work and the specific models that are developed and used.
    \item \xr{Chapter...}
\end{enumerate}